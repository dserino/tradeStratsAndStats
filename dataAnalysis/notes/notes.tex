\documentclass[10pt, a4paper]{article}

\usepackage[english]{babel}
\usepackage[utf8]{inputenc}
\usepackage[hyphens]{url}
\usepackage{amsmath}
\usepackage{graphicx}
\usepackage{listings}
\usepackage{hyperref}
\usepackage{placeins}
\usepackage{mathtools}
\usepackage{verbatim}
\usepackage{tikz}
\usepackage{subcaption}
\usepackage{bm}
\usepackage{titlesec}

\usepackage{algorithm} 
\usepackage{algpseudocode} % algorithmcx 

\setlength\parindent{0pt}

\lstset{
basicstyle=\footnotesize\ttfamily,
breaklines=true,
commentstyle=\color{red},
moredelim=**[is][\slshape]{`}{`},
moredelim=**[is][\color{red}]{@}{@}}

\titleformat*{\section}{\Large\bfseries}
\titleformat*{\subsection}{\large\bfseries}
\titleformat*{\subsubsection}{\normalsize\bfseries}

\newcommand{\blue}{\color{blue}}
\newcommand{\e}{\epsilon}

\newcommand{\todo}[1]{{\color{red}todo: #1}}


\begin{document}

This document is intended for notes on the strategies we are developing.

%%%%%%%%%%%%%%%%%%%%%%%%%%%%%%%%%%%%%%%%%%%%%%%
\section{Numerical Data Techniques}

\subsection{DMD}

\subsection{PCA}

\subsection{Approximate Difference Scheme (ADS)}

Consider a time series $f_j = f(t_j)$ for uniformly spaced
grid, $t_j = j \Delta t + t_0,$ $j = 0, \dots, N_t.$ We wish to obtain
$f_j$ for $j > N_t.$ We suppose that the time series approximately obeys
the difference equation:
\begin{align}
a_0 f_j + a_1 f_{j-1} + ... + a_{M} f_{j-M} \approx 0, \quad j = M, \dots, N_t.
\end{align}
%
Let $\textbf{f}_j = [f_j, f_{j-1}, \dots, f_{j-M}],$ then we want to find a vector 
$\textbf{a}$ approximately orthogonal to $\textbf{f}_j$ for $j = M, \dots, N_t.$
Combine into a matrix equation
\begin{align}
\textbf{F} \textbf{a} = 
[\textbf{f}_M, \textbf{f}_{M+1}, \dots \textbf{f}_{N_t}] \textbf{a} \approx 0.
\end{align}
%
Take the SVD of $\textbf{F},$
\begin{align}
\textbf{F} = \textbf{U} \boldsymbol{\Sigma} \textbf{V}^*.
\end{align}
% 
Since
\begin{align}
\textbf{F} \textbf{v}_i = \sigma_i \textbf{u}_i,
\end{align}
%
the vector that is most closely in the null space of $\textbf{F}$ is 
the last column of $\textbf{V},$ which corresponds to the smallest singular value. 
The smaller the singular value, the better the approximation is. In a scenario 
like the stock market, we will know a piece of information for
each day and may want to predict what will happen to the information during the next day.
Let $\hat{f}$ be the predicted values. The prediction for the next day, $j = N_t+1,$ is
given by
\begin{align}
\hat{f}_{j} = -\frac{1}{a_0}\sum_{i = 1}^{M} a_i f_{j-i}.
\end{align}
%
See \lstinline{approximateDifferenceSchemeTest.m}.

%%%%%%%%%%%%%%%%%%%%%%%%%%%%%%%%%%%%%%%%%%%%%%%
\section{Stock Price Prediction Techniques}

\subsection{Smooth-Extrapolate-Unsmooth (SEUS)}

\begin{enumerate}
\item Stock data is very noisy, first smooth the data so our algorithms will work better.
For example use a trailing average
\begin{align}
\tilde{f}_{j}^M = \frac{1}{M} \sum_{i=0}^{M-1} f_{j-i},
\end{align}
where $\tilde{f}$ is the smoothed variable.
%
\item Use a predictive algorithm to extrapolate $\tilde{f}_{j+1}$ (ADS).
%
\item If possible, find $f_{j+1}$ given $\tilde{f}_{j+1}.$ One quantity of 
interest is the predicted percent increase, given by
\begin{align}
\hat{p}_{j+1} = \frac{\hat{f}_{j+1}-f_j}{f_j}
\end{align}
%
\end{enumerate}

See \lstinline{analyzeStock.m}

\section{Validation Metrics}

\subsection{Win Loss Ratio}
The win-loss ratio is calculated by
\begin{align}
w/l = \left(\sum_{j, \; p_j > 0} p_j\right)\bigg{/}\left(\sum_{j, \; p_j < 0} p_j\right).
\end{align}
%
This is for the simplified scenario where the investor sells the day after a buy and buys
the same principle amount each time. If the win loss ratio is 2, then 
the investor looses a dollar on every 2 dollars of profit. The higher the win-loss ratio,
the more confident the investor will be in the strategy. For stocks that are held for a long 
time, a typical win-loss ratio may be around 1.5 due to high volatility. 


\end{document}